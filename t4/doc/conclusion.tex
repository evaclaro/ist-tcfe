\newpage
\section{Conclusion}
\label{sec:conclusion}

\par The objective of this laboratory assignment was to develop an audio amplifier circuit and the main goal was achieved. However by observing analysis and simulation results it can be seen a difference between the two. This is the resukt of using a non-linear circuit whereas the model used by NGSpice is far more complex than the theoretical model used. Regarding this one and despite the differences, the theoretical model gives good results and can be used when there is no simulation tools to use or to quickly confirmed the simulation results obtained.
\par







The objective of this laboratory assignment was to develop an audio amplifier circuit and the main goal was achieved. However it was achieved not having the best merit. The merit of the circuit was obtained by trial and error, a method that is not perfect and does not result in the best possible results. In this way, we concluded that in order to obtain good results, we were obliged to "yield" part of the merit.

We also note that this time, the results were not equal and exactly the same comparing both NGSpice and Octave.

However, we believe that the differences are not that significant and they can be explained by how NGSpice solves the circuit compared to how it was done in the theoretical analysis, processes that were also explanied on our lectures. To solve the circuit, NGSpice used far more advanced simulation methods for the diodes, with many more parameters, while we used an approximated model with $V_{on}$ and an incremental resistor. 

The error obtained between the average theoretical value and average simulated value is 4.86\% which wouldn't be signficant in a real life cenario but for a online simulation is a bit significant.

This way, the objective should have never been to have equal results, but rather, have results that seemed reasonables, which we believe it was achieved. The merit obtained was 1.538149e-01.

