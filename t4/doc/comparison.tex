\section{Side by Side Comparison}
\label{sec:comparison}

After ending both simulation and theoretical analysis processes, the results were presented on their sections. However, for presenting a prudent interpretation of the result both tables were put side by side.

\subsection{Envelope detector}

The simulated results of the envelope detector output obtained in NGSpice are compared to the theoretical results from Octave as shown below in Figures \ref{fig:sim41} and \ref{fig:vout_env}, respectively.

\begin{figure} [ht]
\centering
\begin{minipage}{.5\textwidth}
  \centering
  \includegraphics[width=0.9\linewidth]{../sim/sim41.pdf}
  \captionof{figure}{Simulated Envelope detector voltage output}
  \label{fig:sim41}
\end{minipage}%
\begin{minipage}{.5\textwidth}
  \centering
  \includegraphics[width=0.9\linewidth]{vout_env.eps}
  \captionof{figure}{Theoretical Envelope detector voltage output}
  \label{fig:vout_env}
\end{minipage}
\end{figure}

The theoretical ripple in the envelope detector is considerably smaller than the simulated one, due to the approximations made in the theoretical diode model.

\begin{figure}
\centering
\begin{minipage}{.5\textwidth}
  \centering
  \includegraphics[width=0.8\linewidth]{../sim/sim42.pdf}
  \captionof{figure}{Simulated voltage output error.}
  \label{fig:sim_error}
\end{minipage}%
\begin{minipage}{.5\textwidth}
  \centering
  \includegraphics[width=0.8\linewidth]{vout_rect.eps}
  \captionof{figure}{Theoritical voltage output error.}
  \label{fig:theo_error}
\end{minipage}
\end{figure}

\subsection{Output Voltage}
The figures 13 and 14 show the comparison between simulated and theoretical values for the voltage transformer being evident there is no difference between the two plots.
The simulated results of the voltage output are compared to the theoretical results as shown below in Figures \ref{fig:sim_vout} and \ref{fig:theo_vout}, respectively.

\begin{figure} [h]
\centering
\begin{minipage}{.5\textwidth}
  \centering
  \includegraphics[width=0.9\linewidth]{../sim/sim43.pdf}
  \captionof{figure}{Simulated voltage output.}
  \label{fig:sim_vout}
\end{minipage}%
\begin{minipage}{.5\textwidth}
  \centering
  \includegraphics[width=0.9\linewidth]{vout_reg.eps}
  \captionof{figure}{Theoretical voltage output.}
  \label{fig:theo_vout}
\end{minipage}
\end{figure}

Once again, the theoretical ripple is considerably smaller than the simulated one, due to the approximations made in the theoretical diode model.

It is also notable that the theoretical output voltage values are always higher than both 12V and the simulation's values, which are lower than 12V.  

The voltage output errors ($V_{out}$ - 12) of the simulation and the theoretical analysis are shown above in Figures \ref{fig:sim_error} and \ref{fig:theo_error}, respectively.


\begin{figure}
\centering
\begin{minipage}{.5\textwidth}
  \centering
  \includegraphics[width=0.8\linewidth]{../sim/sim44.pdf}
  \captionof{figure}{Simulated voltage output error.}
  \label{fig:sim_error}
\end{minipage}%
\begin{minipage}{.5\textwidth}
  \centering
  \includegraphics[width=0.8\linewidth]{defletion.eps}
  \captionof{figure}{Theoretical voltage output error.}
  \label{fig:theo_error}
\end{minipage}
\end{figure}

\pagebreak The theoretical output voltage error is always positive and higher than the simulation's, which is always negative and bigger in absolute value.
Comparing the Voltage ripple and VDC of the simulated and theoretical analysis in tables \ref{tab:sim} and \ref{tab:tab1}, respectively:

\begin{table}[h]
  \centering
  \begin{tabular}{|l|r|}
    \hline    
    {\bf Name} & {\bf Value [V]} \\ \hline
    \input{../sim/op1_tab}
  \end{tabular}
  \caption{Simulated results. mean(v(4)) is the average outuput voltage and vecmax(v(4))-vecmin(v(4)) is the maximum value of ripple. The last value is the merit of the circuit.}
  
  \label{tab:sim}
\end{table}

\begin{table}[h]
  \centering
  \begin{tabular}{|l|r|}
    \hline    
    {\bf Name} & {\bf Value [V]} \\ \hline
    \input{../mat/octave5_tab}
  \end{tabular}
  \caption{Theoretical values. $V_{DC}$ is the average outuput voltage.}
  \label{tab:tab1}
\end{table}










