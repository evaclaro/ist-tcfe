\section{Theoretical Analysis}
\label{sec:analysis}

\hspace{0,5cm} In this section, the circuit shown in Figure~\ref{fig:circuito} is analysed theoretically, in terms of it's node voltages and mesh currents.

\subsection{Mesh Analysis Method}

\hspace{0,5cm} 
For this mesh metod we defined circular currents in the counter-clockwise direction and then the circuit os evaluated considering those new currents.

Starting by number the nodes arbitrarily, assigning current names and directions to all branches also arbitrarily and defining one node as ground (GND) 

Being mesh A the one with the resistors R1, R3 and R4, and the voltage source Va, the mesh B with the resistors R2, R3 and R5 and the voltage controlled current source Ib, the mesh C with the resistors R4, R6 and R7, and the current controlled voltage source Vc, and, at last, the mesh D with the resistor R5, the current source Id and the current controlled voltage source Vc. 

A system of equations obtained applying the Kirchhoff Current Law (KCL)to each mesh can be written as 

\begin{equation}
\begin{bmatrix} 
R_1 + R_3 + R_4 & -R_3 & -R_4 \\ 
-R_4 & 0 & R_4 + R_6 + R_7 - K_C\\
-K_B R_3 & K_B R_3 - 1 & 0
\end{bmatrix} 
\begin{bmatrix} 
I_A \\ 
I_B \\ 
I_C
\end{bmatrix} =
\begin{bmatrix} 
-V_A \\ 
0 \\ 
0
\end{bmatrix}
\end{equation}

It was used 3 equations (4 meshes - 1 = 3 linearly independent equations) : Mesh A, Mesh C and an addicional equation which is $I_b = K_b V_b \textit{where} V_b = (I_B - I_A)R_3$ 

It's important to notice that D loop is independent from the restant ones so we don't need to determine the current $I_D$ as it is given on the inicial data.

Also we didn't use the mesh B because it had a independent current source so we couldn't aply the KVL.

After solving the system with Octave tools we get the Table~\ref{tab:mesh} results.

\begin{table}[h]
  \centering
  \begin{tabular}{|l|r|}
    \hline    
    {\bf Name} & {\bf Value [A or V]} \\ \hline
    IA &   -2.440917089113763e-04\\ \hline
IB &   -2.554476012603903e-04\\ \hline
IC &    9.804366878292505e-04\\ \hline
ID &    1.014556835690000e-03\\ \hline
Ib &   -2.554476012603903e-04\\ \hline
Ic &    9.804366878292505e-04\\ \hline
Vb &   -3.572796600675846e-02\\ \hline
Vc &    7.966966194871432e+00\\ \hline

  \end{tabular}
  \caption{ Results obtained by mesh analysis method with octave tool}
  \label{tab:mesh}
\end{table}


\subsection{Nodal Analysis Method}

\hspace{0,5cm} The Nodal Analysis Method is another general procedure analysing circuits using node voltages as the circuit variables. 

To find the nodal voltages we chose 7 equations (8 nodes - 1 = 7 linearly independent equations) that comprise:
\begin{itemize}
\item  KCL in nodes not connected to voltage sources;
\item Additional equations for nodes related by voltage sources.
\end{itemize}

It was used the equations regarding the nodes 0, 2, 5, 6 therefore it was necessary tree additional equations.

We chose to put the ground zero between tree branches ccorresponding to the ones with $R_1, R_2$ and $R_3$ because it will facilitate the system of equations .

the first equation V1-V4=Va was used for node 1 because node 1 and node 4 are connected to a independent voltage source

secondly knowing that Vc=Kc*Ic and Vc=V3-V7  it was concluded that for node 7 tha equation obtained was V3-V7=Kc(V3-V6)*G6

Finally node 3 is connected to 4 branches so applying  Ohms law to the 3 resistors and knowing that the current that passes through Vc is I4=-Id+(V6-V7)G7 we get the final and thrid equation tha we need: (V4-V3)G4 + (V0-V3)G3 +(V5-V3)G5-Id+(V6-V7)G7

The system of equations that will be solved is:

\begin{equation}
\begin{bmatrix} 
1 & 0 & 0 & 0 & 0 & 0 & 0 & 0 \\
-G_1 - G_2 - G_3 & G_1 & G_2 & G_3 & 0 & 0 & 0 & 0 \\
K_b + G_2 & 0 & -G_2 & -K_b & 0 & 0 & 0 & 0 \\
0 & 1 & 0 & 0 & -1 & 0 & 0 & 0 \\
-K_b & 0 & 0 & K_b + G_5 & 0 & -G_5 & 0 & 0 \\
0 & 0 & 0 & 0 & G_6 & 0 & -G_6 - G_7 & G_7 \\
0 & 0 & 0 & 1 & -K_c G_6 & 0 & K_c G_6 & -1 \\
G_3 & 0 & 0 & -G_4 - G_3 - G_5 & G_4 & G_5 & G_7 & -G_7
\end{bmatrix} 
\begin{bmatrix} 
V_0 \\ 
V_1 \\ 
V_2 \\ 
V_3 \\ 
V_4 \\ 
V_5 \\ 
V_6 \\ 
V_7
\end{bmatrix} =
\begin{bmatrix} 
0\\ 
0 \\ 
0 \\ 
V_a \\
-I_d \\ 
0 \\ 
0 \\ 
I_d
\end{bmatrix}
\end{equation}


After solving the system with Octave tools we get the Table~\ref{tab:nodal} results.

\begin{table}[h]
  \centering
  \begin{tabular}{|l|r|}
    \hline    
    {\bf Name} & {\bf Value [A or V]} \\ \hline
    V0 &    0.000000000000000e+00\\ \hline
V1 &    2.524677350223437e-01\\ \hline
V2 &   -5.181833543548797e-01\\ \hline
V3 &    3.572796600675687e-02\\ \hline
V4 &   -4.904491610977656e+00\\ \hline
V5 &    4.000301352954461e+00\\ \hline
V6 &   -6.935136583530766e+00\\ \hline
V7 &   -7.931238228864721e+00\\ \hline
Vb &   -3.572796600675687e-02\\ \hline
Ib &   -2.554476012603790e-04\\ \hline
Vc &    7.966966194871478e+00\\ \hline
Ic &    9.804366878292561e-04\\ \hline

  \end{tabular}
  \caption{Results obtained by nodal analysis method with octave tool}
  \label{tab:nodal}
\end{table}



