\section{Theoretical Analysis}
\label{sec:analysis}

\hspace{0,5cm} In this section, the  circuit shown in Figure~\ref{fig:circuito} will be analysed theoretically.
The constants used for the resistors and capacitors can be seen bellow.

\begin{center}

$R_1 = 1 k\Omega$

$R_2 = 1 k\Omega$
 
$R_3 = 100 k\Omega $

$R_4 = 1 k\Omega $ 

$C_1 = 220 nF $

$C_2 = 110 nF $

\end{center}

\par In order to fully understand the analysis that will be made, it is necessary to bear in mind that there are three stages in this circuit: the High Pass Stage, the Amplication Stage and the Low Pass Stage. 
\par The first one lets the high frequency sinals pass, and cuts the ones with lower frequencies.It is made with $C_1$ and $R_1$. 
\par The second stage is used for amplify the sinal, being its main component the Operational Amplifier. It's easy to understand that it's on this stage that the gain is maximum. The components of this stage are the OP-AMP, $R_3$ and $R_4$. 
\par The third stage, works essentially for cutting high frequency sinals and letting the lower ones pass. It is made by the remaining components, $R_2$ and $C_2$. 
\par All in all, we build a Band Pass Filter by cuting the higher and the lower frequencies and letting pass a specific band of sinals.

\begin{table}[H]
  \centering
  \begin{tabular}{|l|r|}
    \hline    
    {\bf Name} & {\bf Value [Ohm/F]} \\ \hline
    \input{../mat/octave1_tab}
  \end{tabular}
  \caption{TITULO}
  \label{tab:mat1}
\end{table}

\begin{table}[H]
  \centering
  \begin{tabular}{|l|r|}
    \hline    
    {\bf Name} & {\bf Value [Ohm/F]} \\ \hline
    \input{../mat/octave2_tab}
  \end{tabular}
  \caption{TITULO}
  \label{tab:mat2}
\end{table}

\begin{figure}[H] \centering
\includegraphics[width=0.5\linewidth]{../mat/octave1.eps}
\caption{TITULO}
\label{fig:mat3}
\end{figure}

\begin{figure}[H] \centering
\includegraphics[width=0.5\linewidth]{../mat/octave2.eps}
\caption{TITULO}
\label{fig:mat4}
\end{figure}






