\section{Simulation Analysis}
\label{sec:simulation}

\hspace{0,5cm} This section covers the bandpass filter simulation (using OP-AMP) in NGSpice. As asked we started by using the provided model of the OP-AMP and then the circuit was improved by doing incremental modifications with suitable parameters. In the next tables it is presented the values asked in the lab assignment: output voltage gain, the central frequency and the input and output impedances at this frequency.

\begin{table}[!ht]
  \centering
  \begin{tabular}{|l|r|}
    \hline    
    {\bf Name} & {\bf Value } \\ \hline
    \input{../sim/gain_frequency_tab}
  \end{tabular}
  \caption{Gain and Central Frequency}
  \label{tab:ng3}
\end{table}

\par Following with the analysis, below it is presentend the output and input impedances.

\begin{table}[!ht]
  \centering
  \begin{tabular}{|l|r|}
    \hline    
    {\bf Name} & {\bf Value } \\ \hline
    \input{../sim/input_impedance_tab}
  \end{tabular}
  \caption{Input Impedance}
  \label{tab:ng4}
\end{table}

\begin{table}[!ht]
  \centering
  \begin{tabular}{|l|r|}
    \hline    
    {\bf Name} & {\bf Value } \\ \hline
    \input{../sim/output_impedance_tab}
  \end{tabular}
  \caption{Output Impedance}
  \label{tab:ng4}
\end{table}

\par The graphics below show the frequence response of the output voltage gain in the passband.

\begin{figure}[H] \centering
\includegraphics[width=0.5\linewidth]{../sim/voutph.pdf}
\caption{Simulation gain response in phase degree}
\label{fig:ng1}
\end{figure}

\begin{figure}[H] \centering
\includegraphics[width=0.5\linewidth]{../sim/voutdb.pdf}
\caption{Simulation gain response in dB}
\label{fig:ng6}
\end{figure}

\par Finally, the merit obtained by the group is presentend in the following table. It can be consider that the results were good.

\begin{table}[!ht]
  \centering
  \begin{tabular}{|l|r|}
    \hline    
    {\bf Name} & {\bf Value } \\ \hline
    \input{../sim/deviation_tab}
  \end{tabular}
  \caption{Gain and frequency deviation}
  \label{tab:ng2}
\end{table}

\begin{table}[!ht]
  \centering
  \begin{tabular}{|l|r|}
    \hline    
    {\bf Name} & {\bf Value} \\ \hline
    \input{../sim/merit_tab}
  \end{tabular}
  \caption{Merit}
  \label{tab:ng5}
\end{table}


