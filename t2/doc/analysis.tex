\section{Theoretical Analysis}
\label{sec:analysis}

\hspace{0,5cm} In this section, the circuit shown in Figure~\ref{fig:circuito} is analysed theoretically, in terms of its node voltages and mesh currents.

For this mesh method, circular currents are defined in the counter-clockwise direction and then the circuit is evaluated considering those new currents.

Starting by number the nodes arbitrarily, assigning current names and directions to all branches also arbitrarily and defining one node as ground (GND). 

Being mesh A the one with the resistors $R_1$, $R_3$ and $R_4$, and the voltage source $V_a$, the mesh B with the resistors $R_2$, $R_3$ and $R_5$ and the voltage controlled current source $I_b$, the mesh C with the resistors $R_4$, $R_6$ and $R_7$, and the current controlled voltage source $V_c$, and, at last, the mesh D with the resistor $R_5$, the current source $I_d$ and the current controlled voltage source $V_c$. 

A system of equations obtained applying the Kirchhoff Current Law (KCL)to each mesh can be written as 


It was used 3 equations (4 meshes - 1 = 3 linearly independent equations): Mesh A, Mesh C and an addicional equation which is 

\begin{equation}
I_b = K_b V_b \textrm{, where } V_b = (I_B - I_A)R_3
\end{equation}

It's important to notice that D loop is independent of the remaining ones so it isn't don't need to determine the current $I_D$ as it is given on the initial data.

%After solving the system with Octave tools we get the Table~\ref{tab:mesh} results. In Table~\ref{tab:mesh} voltage values are identified with V and their measure is $V$(Volts), the remaining ones are current values so their units are $A$(Amperes).


The Nodal Analysis Method is another general procedure for analysing circuits using node voltages as the circuit variables. 

To find the nodal voltages we chose 7 equations (8 nodes - 1 = 7 linearly independent equations) that comprise:
\begin{itemize}
\item  KCL in nodes not connected to voltage sources;
\item Additional equations for nodes related by voltage sources.
\end{itemize}

It was used the equations regarding the nodes 0, 2, 5, 6 therefore it was necessary three additional equations.

We chose to put the ground zero between three branches corresponding to the ones with $R_1, R_2$ and $R_3$ because it will facilitate the system of equations.

The next equation was used for node 1 because node 1 and node 4 are connected to an independent voltage source.
\begin{equation}
V_1-V_4=V_a 
\end{equation}

Secondly knowing that $V_c=K_c I_c$ and $V_c=V_3 - V_7$  it was concluded that for node 7 the equation obtained was

\begin{equation}
V_3 - V_7 = K_c (V_3 - V_6) G_6
\end{equation}

To find the third equation (Equation~\ref{eq:super}) it was used the continuity of the current to create a "super-knot" (nodes 3 and 7)

\begin{equation}
(V_4-V_3)G_4 + (V_0-V_3)G_3 +(V_5-V_3)G_5-I_d+(V_6-V_7)G_7=0
\label{eq:super}
\end{equation}



\subsection{Node analysis for t$<$0}

The system of equations that will be solved is:

\begin{centrar}
\begin{bmatrix} 
1 & 0 & 0 & 0 & 0 & 0 & 0 & 0 \\
0 & 1 & 0 & 0 & 0 & 0 & 0 & 0 \\
0 & G_1 & -G_1 -G_2 -G_3 & G_2 & G_3 & 0 & 0 & 0 \\
0 & 0 & K_b + G_2 & -G_2 & -K_b & 0 & 0 & 0 \\
0 & -G_1 & G_1 & 0 & G_4 & 0 & G_6 & 0 \\
0 & 0 & K_b & 0 & -K_b - G_5 & G_5 & 0 & 0 \\
0 & 0 & 0 & 0 & 0 & 0 & G_6 - G_7 & G_7 \\
0 & 0 & 0 & 0 & 1 & 0 & K_d G_6 & -1
\end{bmatrix} 
\begin{bmatrix} 
V_0i \\ 
V_1i \\ 
V_2i \\ 
V_3i \\  
V_5i \\ 
V_6i \\ 
V_7i\\
V_8i
\end{bmatrix} =
\begin{bmatrix} 
0\\ 
V_s \\ 
0 \\ 
0\\
0\\ 
0 \\ 
0 \\ 
0
\end{bmatrix}
\end{centrar}



\begin{table}[H]
  \centering
  \begin{tabular}{|l|r|}
    \hline    
    {\bf Name} & {\bf Value [A or V]} \\ \hline
    \input{../mat/octave1i_tab}
  \end{tabular}
  \caption{LEGENDA}
  \label{tab:mat1}
\end{table}


\begin{table}[H]
  \centering
  \begin{tabular}{|l|r|}
    \hline    
    {\bf Name} & {\bf Value [A or V]} \\ \hline
    \input{../mat/octave1v_tab}
  \end{tabular}
  \caption{LEGENDA}
  \label{tab:mat1a}
\end{table}



\subsection{Determining $R_{eq}$}

The system of equations that will be solved is:

\begin{centrar}
\begin{bmatrix} 
1 & 0 & 0 & 0 & 0 & 0 & 0 & 0 \\
0 & 1 & 0 & 0 & 0 & 0 & 0 & 0 \\
0 & G_1 & -G_1 -G_2 -G_3 & G_2 & G_3 & 0 & 0 & 0 \\
0 & 0 & K_b + G_2 & -G_2 & -K_b & 0 & 0 & 0 \\
0 & -G_1 & G_1 & 0 & G_4 & 0 & G_6 & 0 \\
0 & 0 & 0 & 0 & 0 & 1 & 0 & -1 \\
0 & 0 & 0 & 0 & 0 & 0 & G_6 - G_7 & G_7 \\
0 & 0 & 0 & 0 & 1 & 0 & K_d G_6 & -1
\end{bmatrix} 
\begin{bmatrix} 
V_0t0 \\ 
V_1t0  \\ 
V_2t0 \\ 
V_3t0 \\  
V_5t0 \\ 
V_6t0 \\ 
V_7t0\\
V_8t0
\end{bmatrix} =
\begin{bmatrix} 
0\\ 
0 \\ 
0 \\ 
0\\
0\\ 
V_x \\ 
0 \\ 
0
\end{bmatrix}
\end{centrar}


\begin{table}[H]
  \centering
  \begin{tabular}{|l|r|}
    \hline    
    {\bf Name} & {\bf Value [A or V]} \\ \hline
    \input{../mat/octave2v_tab}
  \end{tabular}
  \caption{LEGENDA}
  \label{tab:mat2}
\end{table}

\begin{table}[H]
  \centering
  \begin{tabular}{|l|r|}
    \hline    
    {\bf Name} & {\bf Value [A or V]} \\ \hline
    \input{../mat/octave2i_tab}
  \end{tabular}
  \caption{LEGENDA}
  \label{tab:mat2a}
\end{table}

\begin{table}[H]
  \centering
  \begin{tabular}{|l|r|}
    \hline    
    {\bf Name} & {\bf Value [A or V]} \\ \hline
    \input{../mat/octave2valores_tab}
  \end{tabular}
  \caption{LEGENDA}
  \label{tab:mat2b}
\end{table}

\subsection{Natural Solution with node analysis for t$\geq$0}

\begin{figure}[H] \centering
\includegraphics[width=0.5\linewidth]{../mat/natural.eps}
\caption{LEGENDA} %mudar legendaaaaaaa
\label{fig:mat3}
\end{figure}

\subsection{Forced Solution with node analysis for t$\geq$0}

\begin{centrar}
\begin{bmatrix} 
-G_1 & G_1 +G_2+G_3 & -G_2 & -G_3 & 0 & 0 & 0  \\
0 & -G_2 -K_b & G_2 & K_b & 0 & 0 & 0 \\
0 & K_b & 0 & K_b -G_5 & G_5+ 1/Z_c & 0 & -1/Z_c \\
0 & 0 & 0 & 0 & 0 & G_6+G_7 & -G_7 \\
1 & 0 & 0 & 0 & 0 & 0 & 0 & 0 \\
0 & 0 & 0 & 1 & 0 & K_d G_6 & -1 \\
G_1 & -G_1 & 0 & -G_4 & 0 & -G_6 & 0 
\end{bmatrix} 
\begin{bmatrix} 
V_0 \\ 
V_1  \\ 
V_2 \\ 
V_3 \\  
V_5 \\ 
V_6 \\ 
V_7\\
V_8
\end{bmatrix} =
\begin{bmatrix} 
0\\ 
0 \\ 
0 \\ 
0\\
1\\ 
0 \\  
0
\end{bmatrix}
\end{centrar}



\begin{table}[H]
  \centering
  \begin{tabular}{|l|r|}
    \hline    
    {\bf Name} & {\bf Value [A or V]} \\ \hline
    \input{../mat/octave4p_tab}
  \end{tabular}
  \caption{LEGENDA}
  \label{tab:mat4}
\end{table}

\begin{table}[H]
  \centering
  \begin{tabular}{|l|r|}
    \hline    
    {\bf Name} & {\bf Value [A or V]} \\ \hline
    \input{../mat/octave4v_tab}
  \end{tabular}
  \caption{LEGENDA}
  \label{tab:mat4a}
\end{table}



\subsection{Ponto 5}

\begin{figure}[H] \centering
\includegraphics[width=0.5\linewidth]{../mat/theoretical_5.eps}
\caption{LEGENDA} %mudar legendaaaaaaa
\label{fig:mat4}
\end{figure}


\subsection{Frequency Responses}

\begin{figure}[H] \centering
\includegraphics[width=0.5\linewidth]{../mat/Phase(degrees).eps}
\caption{LEGENDA} %mudar legendaaaaaaa
\label{fig:mat5db}
\end{figure}

\begin{figure}[H] \centering
\includegraphics[width=0.5\linewidth]{../mat/MagnitudedB.eps}
\caption{LEGENDA} %mudar legendaaaaaaa
\label{fig:mat5ps}
\end{figure}



After solving the system with Octave tools we get the Table results.In Table voltage values are identified with V and their measure is $V$(Volts), the remaining ones are current values so their units are $A$(Amperes).





