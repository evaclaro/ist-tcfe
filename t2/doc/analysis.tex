\section{Theoretical Analysis}
\label{sec:analysis}

\hspace{0,5cm} In this section, the circuit shown in Figure~\ref{fig:circuito} is analysed theoretically, with the Nodal Analysis Method, which uses node voltages as the circuit variables. 


\subsection{Node analysis for t$<$0}

\hspace{0,5cm} The aim of using this method is to determine every node voltage, therefore we considered the node 0 a reference node. However, due to the existence of the independent voltage source $V_s$ and the current controlled voltage source $V_d$, it is useless to analyse nodes 0 and 1 (connected to $V_s$) and also nodes 5 and 8 (connected to $V_d$) since nodes connected to voltage sources can't be analysed. So, this means that we can only analyse nodes 2,3,6 and 7.

In order to determine all the unknown node voltage values, it is necessary to have eight linearly independent equations. Before $t$ = 0s, $v_s$ is constant, which means that the capacitor is assumed to also be constant and fully charged, behaving like an open-circuit, therefore $I_c$ = 0.

Four of the needed equations are given by the nodal analysis:
\par
Node 2
\begin{equation}
(V_3-V_2)G_2 - (V_2-V_5)G_3 - (V_2-V_1)G_1 = 0
\label{eq:1.1}
\end{equation}

Node 3
\begin{equation}
(V_2-V_5)K_b - (V_3-V_2)G_2 = 0
\label{eq:1.2}
\end{equation}

Node 6
\begin{equation}
(V_2-V_5)K_b - (V_6-V_5)G_5 = 0
\label{eq:1.3}
\end{equation}

Node 7
\begin{equation}
-V_7 G_6 - (V_7-V_8)G_8 = 0
\label{eq:1.4}
\end{equation}

We still need another four equations. For this reason, we can use these two trivial equations
\begin{equation}
V_0=0
\label{eq:1.5}
\end{equation}
\begin{equation}
V_1=V_s
\label{eq:1.6}
\end{equation}

We can also use the fact that $V_5 - V_8 = V_d = K_d I_d$ and $I_d$, according the Ohm's Law, is equal to $G_7 (V_0 - V_7)$ which means
\begin{equation}
-V_7 G_6 K_d = V_5 - V_8
\label{eq:1.7}
\end{equation}

At last, since there was still missing an equation, we considered a super node containing the branch that includes $v_s$
\begin{equation}
(V_2-V_1)G_1 + V_5 G_4 +  V_7 G_6 = 0
\label{eq:1.8}
\end{equation}

The system of equations that will be solved in form of matrix and with the assitance of Octave is:
\begin{centrar}
\begin{bmatrix} 
1 & 0 & 0 & 0 & 0 & 0 & 0 & 0 \\
0 & 1 & 0 & 0 & 0 & 0 & 0 & 0 \\
0 & G_1 & -G_1 -G_2 -G_3 & G_2 & G_3 & 0 & 0 & 0 \\
0 & 0 & K_b + G_2 & -G_2 & -K_b & 0 & 0 & 0 \\
0 & -G_1 & G_1 & 0 & G_4 & 0 & G_6 & 0 \\
0 & 0 & K_b & 0 & -K_b - G_5 & G_5 & 0 & 0 \\
0 & 0 & 0 & 0 & 0 & 0 & G_6 - G_7 & G_7 \\
0 & 0 & 0 & 0 & 1 & 0 & K_d G_6 & -1
\end{bmatrix} 
\begin{bmatrix} 
V_0i \\ 
V_1i \\ 
V_2i \\ 
V_3i \\  
V_5i \\ 
V_6i \\ 
V_7i\\
V_8i
\end{bmatrix} =
\begin{bmatrix} 
0\\ 
V_s \\ 
0 \\ 
0\\
0\\ 
0 \\ 
0 \\ 
0
\end{bmatrix}
\end{centrar}

\begin{table}[H]
  \centering
  \begin{tabular}{|l|r|}
    \hline    
    {\bf Name} & {\bf Value [V]} \\ \hline
    \input{../mat/octave1v_tab}
  \end{tabular}
  \caption{Nodal voltage values (t$<$0)}
  \label{tab:mat1a}
\end{table}

By using Ohm's Law, we calculate the values of the currents passing in the resistors. The value of the current in $v_s$ is simmetrical to $I_R1$ and the value of the current in $V_d$ is simmetrical to $I_{R6}$.
\begin{table}[H]
  \centering
  \begin{tabular}{|l|r|}
    \hline    
    {\bf Name} & {\bf Value [A or V]} \\ \hline
    \input{../mat/octave1i_tab}
  \end{tabular}
  \caption{Branch and voltage sources current values (t$<$0)}
  \label{tab:mat1}
\end{table}



\subsection{Determining $R_{eq}$}

Analysing the circuit now for t = 0, we notice that $v_s$ equals 0 (short circuit), thus $V_1$ is also null.
In order to determine the equivelant resistence ($R_{eq}$)of the circuit seen from the capacitors terminals and the time constant (much needed for the following subsections), we replace the capacitor with a voltage source $V_x = V_6 - V_8$ and do another nodal analysis to determine the current supplied by $V_x$, which will be called $I_x$.
These two values determine $R_{eq}$ with the equation:
The system of equations that will be solved is:
\begin{equation}
R_{eq} = V_x/I_x
\label{eq:1.9}
\end{equation}

We use the values $V_6$ and $V_8$ from the previous subsection since the voltage drop at the terminals of the capacitor needs to be a continuous function, this means that there cannot be a sudden energy discontinuity in the capacitor.
This is the most efficient procedure to determine the equivelant resistence in such a complex circuit, with this reasoning being based on the usage of the $Thevenin$ and $Norton$ theorems, where $V_x$ is equivalent to $Thevenin's$ voltage and $I_x$ is $Norton's$ current.

For this nodal analysis, we used almost the same equations that were used in the previous subsection, except, of course, the trivial one for $V_1$, since now $V_1 = 0$, and we replaced the node 6 equation with the new equation $V_x = V_6 - V_8$.

The system of equations that will be solved in form of matrix and with the assitance of Octave is:
\begin{centrar}
\begin{bmatrix} 
1 & 0 & 0 & 0 & 0 & 0 & 0 & 0 \\
0 & 1 & 0 & 0 & 0 & 0 & 0 & 0 \\
0 & G_1 & -G_1 -G_2 -G_3 & G_2 & G_3 & 0 & 0 & 0 \\
0 & 0 & K_b + G_2 & -G_2 & -K_b & 0 & 0 & 0 \\
0 & -G_1 & G_1 & 0 & G_4 & 0 & G_6 & 0 \\
0 & 0 & 0 & 0 & 0 & 1 & 0 & -1 \\
0 & 0 & 0 & 0 & 0 & 0 & G_6 - G_7 & G_7 \\
0 & 0 & 0 & 0 & 1 & 0 & K_d G_6 & -1
\end{bmatrix} 
\begin{bmatrix} 
V_0t0 \\ 
V_1t0  \\ 
V_2t0 \\ 
V_3t0 \\  
V_5t0 \\ 
V_6t0 \\ 
V_7t0\\
V_8t0
\end{bmatrix} =
\begin{bmatrix} 
0\\ 
0 \\ 
0 \\ 
0\\
0\\ 
V_x \\ 
0 \\ 
0
\end{bmatrix}
\end{centrar}

\begin{table}[H]
  \centering
  \begin{tabular}{|l|r|}
    \hline    
    {\bf Name} & {\bf Value [V]} \\ \hline
    \input{../mat/octave2v_tab}
  \end{tabular}
  \caption{Nodal voltage values (t$<$0)}
  \label{tab:mat2}
\end{table}

By using Ohm's Law, we calculate the values of the currents passing in the resistors.
\begin{table}[H]
  \centering
  \begin{tabular}{|l|r|}
    \hline    
    {\bf Name} & {\bf Value [A]} \\ \hline
    \input{../mat/octave2i_tab}
  \end{tabular}
  \caption{Branch and voltage sources current values (t=0)}
  \label{tab:mat2a}
\end{table}

Now that we have all values, we can use the Kirchhoff Current Law (KCL) in node 6 in order to compute the value of $I_x$
\begin{equation}
I_x = -K_b (V_2 - V_5) - (V_6 - V_5) G5
\label{eq:1.10}
\end{equation}

With the values of $V_x$ and $I_x$ determined, we can calculate $R_{eq}$ with \ref{eq:1.9} and the time constant value with the equation

\begin{equation}
\tau = R_{eq} C
\label{eq:1.11}
\end{equation}

\begin{table}[H]
  \centering
  \begin{tabular}{|l|r|}
    \hline    
    {\bf Name} & {\bf Value} \\ \hline
    \input{../mat/octave2valores_tab}
  \end{tabular}
  \caption{$I_x$ (in A), $R_{eq}$ (in Ohm) and $tau(\tau)$ (adimensional) values}
  \label{tab:mat2b}
\end{table}

\subsection{Natural Solution with node analysis for t$\geq$ 0}
The aim of this section is to calculate the natural solution of $v_{6n} (t)$. The natural response is what the circuit does including the initial conditions (initial voltage of the capacitor) but with the input surpressed.
Knowing the general solution ($v_{6n} (t) = V_6(+\infty) + (V_6(0) - V_6(+\infty))e^{\frac{-t}{\tau}}$), and the fact that the capacitor begins charged but it discharges as the time passes, since it consumes energy, meaning that $V_6(+\infty) = 0$, we can write the following equation:
\begin{equation}
v_{6n} (t) = V_6(0) e^{\frac{-t}{\tau}}
\label{eq:1.12}
\end{equation}

We also know that $V_x = V_6 - V_8$, and that the value of $V_8 (0)$ is aproximately 0, which means that
\begin{equation}
v_{6n} (t) = V_x e^{\frac{-t}{\tau}}
\label{eq:1.13}
\end{equation}

Hence, the graph of $V_{6n}$ in function of the time, in the inteerval [0;20] ms is represent in \ref{fig:mat3}. The result is no suprise, as it shows below, being a negative exponential graph.

\begin{figure}[H] \centering
\includegraphics[width=0.5\linewidth]{../mat/natural.eps}
\caption{Natural solution $v_{6n} (t)$} 
\label{fig:mat3}
\end{figure}

\subsection{Forced Solution with node analysis for t$\geq$0}

In this Section, we have used a phasor voltage source $V_s$, which was replaced with its complex amplitude 1.

Also considering that,
\begin{equation}
     Z_c = \frac{1}{j\omega c}
\end{equation}\par
Where $\omega= 2\pi f$ and $f= 1000Hz$.\par

We obtain the equation for $I_c$:

\begin{equation}
  I_c=(V_6-V_8)/Z_c.
\end{equation}

For the supernode, we use the following equation:

\begin{equation}
  (V_1-V_2)G_1+(V_4-V_5)G_4+I_d=0.
\end{equation}


Then, replacing C with  impedance $Z_c$ and running the nodal analysis to determine the phasor voltages in all nodes we have the following matrix:\par

\begin{centrar}
\begin{bmatrix} 
-G_1 & G_1 +G_2+G_3 & -G_2 & -G_3 & 0 & 0 & 0  \\
0 & -G_2 -K_b & G_2 & K_b & 0 & 0 & 0 \\
0 & K_b & 0 & K_b -G_5 & G_5+ 1/Z_c & 0 & -1/Z_c \\
0 & 0 & 0 & 0 & 0 & G_6+G_7 & -G_7 \\
1 & 0 & 0 & 0 & 0 & 0 & 0 & 0 \\
0 & 0 & 0 & 1 & 0 & K_d G_6 & -1 \\
G_1 & -G_1 & 0 & -G_4 & 0 & -G_6 & 0 
\end{bmatrix} 
\begin{bmatrix} 
V_0 \\ 
V_1  \\ 
V_2 \\ 
V_3 \\  
V_5 \\ 
V_6 \\ 
V_7\\
V_8
\end{bmatrix} =
\begin{bmatrix} 
0\\ 
0 \\ 
0 \\ 
0\\
1\\ 
0 \\  
0
\end{bmatrix}
\end{centrar}

We have calculated the module of $V_6$ with "abs" Octave's function. Then, the phase is calculated with the "imag", the "real" and the "atan" funcions.The obtained results are the ones that follow next. \par


\begin{table}[H]
  \centering
  \begin{tabular}{|l|r|}
    \hline    
    {\bf Name} & {\bf Value [Rad]} \\ \hline
    \input{../mat/octave4p_tab}
  \end{tabular}
  \caption{Phase Results}
  \label{tab:mat4}
\end{table}

\begin{table}[H]
  \centering
  \begin{tabular}{|l|r|}
    \hline    
    {\bf Name} & {\bf Value [V]} \\ \hline
    \input{../mat/octave4v_tab}
  \end{tabular}
  \caption{Amplitudes Results}
  \label{tab:mat4a}
\end{table}


\subsection{Ponto 5}
In this subsection, we determine the final total solution for the value of $v_6$ for the given frequency of 1000 Hz (= 1kHz) by superimposing the natural and forced solutions (determined in the third and forth subsection, respectively), giving us the following equation:
\begin{equation}
v_6(t) = v_{6n}(t) + v_{6f}(t)
\label{eq:1.20}
\end{equation}
In figure \ref{fig:mat4}, we plotted the graphs of $v_6(t)$ and $v_s(t)$ in the interval [-5;20] ms. We can clearly divide the solutions in three parts: when $t=[-5;0[ ms$, $v_s = V_s$ and $v_6 = V_6i$ (the value we calculated in the first subsection); when $t=0 ms$, $v_s = 0$ and $v_6 = V_6t0$ (the value we calculated in the second subsection); and for $t=]0;20] ms$, $v_s = sin(2\pi f t)$ and $v_6$ is given by the equation \ref{eq:1.20}.

\begin{figure}[H] \centering
\includegraphics[width=0.5\linewidth]{../mat/theoretical_5.eps}
\caption{$v_s(t)$ and the final solution of $v_6(t)$ in the interval [-5;20] ms for the frequency of 1000 Hz}
\label{fig:mat4}
\end{figure}


\subsection{Frequency Responses}
In this subsection, we analised how the complex voltages $v_c$, $v_s$ and $v_6$ are not indepedent of the frequency values, comparing them in the same graphs, with the frequency range being from 0.1 Hz to 1MHz. Just like in the forth subsection, we used the nodal method in order to obtain the values of the voltages that don't depend on the frequency. By knowing these voltages, we can now compute the values $v_6$, $v_8$ and, therefore, $v_c$ and also $v_s$, knowing that the value of this voltage is constant due to its independence from the frequency ($v_s = e^(-i\pi/2)$), so there is no need to do any calculation for the phase.
The values had their angles (that correspond to their phases) converted from radians to degrees, were restricted to the interval [-90;90]º in the computation and the frequencies were in a logaritmic scale. We can see in figure \ref{fig:mat5db} that, as predicted, $v_6$ and $v_c$ have their angles decreasing with the increase of the frequency.

\begin{figure}[H] \centering
\includegraphics[width=0.5\linewidth]{../mat/Phase(degrees).eps}
\caption{Phase of $v_6$, $v_c$ and $v_s$ (in degrees)}
\label{fig:mat5db}
\end{figure}

In order to represent the results, we used the magnitude values in dB (decibles) calculated with the following equation:
\begin{equation}
magnitude = 20log_(10)(|x|)
\label{eq:1.21}
\end{equation}
With x being the phasor voltage value. 
We also put the frequencies here in a logaritmic scale. The amplitude attributted to $v_s$ is 1, which means that the magnitude is 0 ($20log_(10)(1)=0$), and is, as expected, constant (proving that this voltage is independent from the frequency), while the other two voltages decrease when the frequency goes up. This should be no suprise for $v_c$ to decrease because of the impedance of the capacitor
\begin{equation}
Z = -i/(wC)
\label{eq:1.22}
\end{equation}

\begin{figure}[H] \centering
\includegraphics[width=0.5\linewidth]{../mat/MagnitudedB.eps}
\caption{Magnitude of $v_6$, $v_c$ and $v_s$ (in decibles)}
\label{fig:mat5ps}
\end{figure}



After solving the system with Octave tools we get the Table results.In Table voltage values are identified with V and their measure is $V$(Volts), the remaining ones are current values so their units are $A$(Amperes).





