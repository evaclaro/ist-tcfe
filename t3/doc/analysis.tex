\section{Theoretical Analysis}
\label{sec:analysis}

\hspace{0,5cm} In this section, the circuit shown in Figure~\ref{fig:circuito} is analysed theoretically, with the Nodal Analysis Method, which uses node voltages as the circuit 


\subsection{Natural Solution with node analysis for t$\geq$ 0}
The aim of this section is to calculate the natural solution of $v_{6n} (t)$. 
Hence, the graph of $V_{6n}$ in function of the time, in the interval [0;20] ms is represent in \ref{fig:mat3}. The result is no suprise, as it shows below, being a negative exponential graph.

\begin{figure}[H] \centering
\includegraphics[width=0.5\linewidth]{../mat/defletion.eps}
\caption{Natural solution $v_{6n} (t)$} 
\label{fig:mat3}
\end{figure}


\subsection{Natural and Forced Superimposed}
In this subsection, we determine the final total solution for the value of $v_6$ for the given 

In Figure \ref{fig:mat4} we plotted the graphs of $v_6(t)$ and $v_s(t)$ in the interval [-5;20] ms. We can clearly divide the solutions in three parts:


\begin{figure}[H] \centering
\includegraphics[width=0.5\linewidth]{../mat/vout_reg_env.eps}
\caption{$v_s(t)$ and the final solution of $v_6(t)$ in the interval [-5;20]ms for the frequency of 1000Hz}
\label{fig:mat4}
\end{figure}





